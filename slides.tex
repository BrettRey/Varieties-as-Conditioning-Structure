% Varieties as Conditioning Structure - LVC Presentation
% Brett Reynolds
% March 13, 2026

\documentclass[aspectratio=169,12pt]{beamer}

% Theme - clean, minimal
\usetheme{metropolis}
\setbeamercolor{background canvas}{bg=white}
\setbeamercolor{normal text}{fg=black}
\setbeamercolor{frametitle}{bg=black,fg=white}

% Fonts
\usepackage{fontspec}
\setmainfont{EB Garamond}
\setsansfont{Helvetica Neue}
\setmonofont{Menlo}

% Packages
\usepackage{booktabs}

% Custom commands
\newcommand{\term}[1]{\textsc{#1}}
\newcommand{\mention}[1]{\textit{#1}}

% Remove navigation
\setbeamertemplate{navigation symbols}{}

% Metadata
\title{Varieties as Conditioning Structure}
\author{Brett Reynolds}
\institute{Humber College}
\date{LVC Research Group --- March 13, 2026}

\begin{document}

% ============================================================================
{
\setbeamertemplate{footline}{}
\begin{frame}[noframenumbering]
\vspace{-2em}
\titlepage
\end{frame}
}

% ============================================================================
\begin{frame}{The claim}

\large
Register, dialect, and discourse community aren't three kinds of thing.

\vspace{1em}

They're three different \textbf{anchors} for appropriateness judgments.

\vspace{1.5em}

\normalsize
The framework: varieties as \textbf{conditioning structure} --- stable patterns in what forms are appropriate given what.

\end{frame}

% ============================================================================
\begin{frame}{Two sources}

\textbf{Wiese (2023):} Grammar lives in \term{communicative situations}, not named languages. Varieties describe regularities across situations.

\vspace{1.5em}

\textbf{O'Connor (2019):} Linguistic choices solve \term{coordination problems}. What stabilises is what coordinates, not what clusters naturally.

\vspace{1.5em}

\textbf{Combined:} Varieties are stable partitions of situation-space, shaped by coordination pressures.

\end{frame}

% ============================================================================
\begin{frame}{The three anchors}

\small
\begin{tabular}{@{}lll@{}}
\toprule
\textbf{Anchor} & \textbf{Conditions on} & \textbf{Traditional label}\\
\midrule
\term{Situation} & Activity, setting, goals & Register\\[0.5em]
\term{Ascription} & How you're classified & Dialect\\[0.5em]
\term{Identification} & Whose norms you orient to & Discourse community\\
\bottomrule
\end{tabular}

\vspace{1em}
\normalsize

Same judgment type. Different conditioning variable.

\end{frame}

% ============================================================================
\begin{frame}{Unpacking the anchors}

\textbf{Situation:} Observable context --- what we're doing, where, with what stakes.

\vspace{1em}

\textbf{Ascription:} Social classification by others --- how listeners place you, based on perceived background, history, group membership. Not chosen.

\vspace{1em}

\textbf{Identification:} Normative orientation --- whose approval you seek, whose standards you hold yourself to. Chosen (within constraints).

\end{frame}

% ============================================================================
\begin{frame}{The interesting space}

\large
Ascription and identification can \textbf{align} --- or they can \textbf{diverge}.

\vspace{1em}
\normalsize

When they align: forms appropriate to your ascribed variety also signal your identification. No tension.

\vspace{0.8em}

When they diverge: you're navigating between how you're heard and who you're orienting to.

\vspace{1em}

\textbf{That gap is where style-shifting, accommodation, and ``passing'' live.}

\end{frame}

% ============================================================================
\begin{frame}{A case}

Someone ascribed as an MTE speaker, with friends:

\vspace{0.5em}
\hspace{2em}\mention{Yo, mans made it --- finally done that assignment.}

\vspace{1.5em}

Same person, ten minutes later, with a teacher:

\vspace{0.5em}
\hspace{2em}\mention{Yeah, I finished it this morning.}

\vspace{1.5em}

The situation shifted. But so might identification --- orienting to different norms, seeking different approval.

\end{frame}

% ============================================================================
\begin{frame}{What this reframes}

\begin{tabular}{@{}ll@{}}
\textbf{Old framing} & \textbf{New framing}\\
\midrule
Code-switching between X and Y & Shift in conditioning anchor\\[0.4em]
Style as individual expression & Coordination with a reference group\\[0.4em]
Dialect as property of speaker & Ascription by listeners\\[0.4em]
Register as property of situation & What situation licenses\\
\end{tabular}

\vspace{1em}

The categories don't disappear. They get reanalysed as dimensions of conditioning.

\end{frame}

% ============================================================================
\begin{frame}{Agency}

\large
Ascription is imposed. Identification is (partially) chosen.

\vspace{1em}
\normalsize

The gap between them is a major locus of speaker agency.

\vspace{1em}

Years later, even with close friends, someone might say \mention{I finished} instead of \mention{mans made it}. Same situation, same ascription. Different identification.

\end{frame}

% ============================================================================
\begin{frame}{The dark side}

A teacher hears \mention{mans} and something shifts --- in expectations, evaluation, behaviour.

\vspace{0.8em}

That shift is an appropriateness judgment, conditioned on inferred ascription.

\vspace{0.8em}

When everyone makes similar judgments, they become \textbf{coordination signals}.

\vspace{0.8em}

``Standard'' vs ``non-standard'' = stable equilibrium. Self-sustaining. Hard to exit unilaterally.

\end{frame}

% ============================================================================
\begin{frame}{Not fatalism}

\large
This is a diagnosis, not an excuse.

\vspace{1.5em}
\normalsize

Coordination equilibria are stable, but they're not immutable.

\vspace{1em}

Change requires changing payoffs or changing what's conditioned on --- not just changing attitudes.

\vspace{1em}

The framework makes visible \textit{where} intervention could work.

\end{frame}

% ============================================================================
\begin{frame}{Summary}

\begin{enumerate}
\item Varieties are conditioning structures, not kinds of language
\item Register, dialect, discourse community = three anchors for the same judgment
\item The ascription--identification gap is analytically central
\item Inequity stabilises through coordination, not just prejudice
\end{enumerate}

\vspace{1.5em}

{\small Paper: \texttt{github.com/BrettRey/Varieties-as-Conditioning-Structure}}

\end{frame}

% ============================================================================
\begin{frame}{Discussion}

\large

Does the S/A/I distinction help with phenomena you work on?

\vspace{1em}

Where does it break?

\vspace{1em}

What would operationalising it look like?

\end{frame}

\end{document}
